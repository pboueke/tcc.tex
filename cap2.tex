\section{Sistemas de Aprendizado de Máquina em Ambiente de Produção}

\paragraph{}Sistemas de Aprendizado de Máquina são sistemas que combinam algoritmos e modelos matemáticos da área de Aprendizado de Máquina em soluções que visam a implantação do uso de tais algoritmos e modelos em ambientes reais. Aprendizado de máquina, por sua vez, se refere a área de estudos voltata ao desenvolvimento e compreensão de modelos matemáticos caracterizados pelo aprimoramento de seus resultados por meio da ingestão de dados de treino, de forma que esses modelos possam realizar predições e decisões sem que sejam explicitamente programados para o fazerem, como aponta Christopher Bishop \cite{Bishop}.

\paragraph{}Com o surgimento de equipamentos de alto poder computacional a preços acessíveis, o campo de Aprendizado de Máquina foi capaz de ser remodelado a partir de princípios e modelos que anteriormente não eram praticáveis por conta das limitações técnicas da época. Com essa mudança, o campo se tornou um foco para o surgimento de novas tecnologias que se apresentam, em grande parte, como inovações da academia e de núcleos de pesquisa. O surgimento de novos paradigmas no campo não se viu acompanhado, em mesma escala, pelo surgimento de novos sistemas e metodologias de implantação das novas tecnologias em ambientes não acadêmicos.

\paragraph{}Enquanto que modelos tomadores de decisões dominam a área, sistemas que permitiriam a implantação de tais modelos em ambientes de produção se encontram pouco difundidos. Atualmente, os principais sistemas desse tipo se encontram hospedados em soluções prontas por provedores de infra-estrutura gerenciada, como é o caso do \textit{Amazon SageMaker} e do \textit{Machine Learning Studio}, soluções pagas providenciadas pela \textit{AWS} e pela \textit{Azure}. Em contrapartida, soluções abertas e gratuitas ainda dependem de um grande trabalho de desenvolvimento por parte dos responsáveis pela implantação, requerendo um esforço similar ao desenvolvido no trabalho aqui exposto.

\section{Agendadores de Tarefas}.


\paragraph{}Para a implantação de um sistema de \textit{stream} de dados para processamento constante, é necessária uma forma para a inicialização da execução de programas e tarefas correlatas ao trabalho sendo executado. Nesse sentido, considera-se úteis ferramentas agendadoras de tarefas, ou \textit{job shedulers}. Essas são ferramentas responsáveis pela execução agendada de programas, viabilizando metas de frequencia de processamento e execução.

\subsection{Slurm}

\paragraph{}Uma dessas ferramentas é o programa conhecido por \textit{slurm} \cite{Slurm}, um \textit{workload manager} capaz de realizar o agendamento de tarefas. Essa é uma das principais ferramentas de código aberto utilizada em sistemas \textit{Unix-like}, capaz de funcionar em \textit{clusters} computacionais de forma distribuída ou baseada em apenas um instância. O Slurm não reqier modificações no \textit{kernel} do sistema operacional e é relativamente auto contido.

\paragraph{}As três principais funções do \textit{slurm} são a alocação de recursos de maneira exclusiva e não exclusiva a usuários por um período de tempo; prover um \textit{framework} para iniciar, executar e monitorar tarefas paralelas em um conjunto de nós computacionais; arbitrar a contenção de recursos gerenciando filas de trabalhos pendentes. Sua arquitetura consiste de um daemon rodando em cada nó e um daemon central no nó principal. Os daemons provém comunicação hierárquica tolerante a falhas.

\paragraph{}Os nós da rede se dividem em partições, cada uma sendo uma fila de tarefas, cada tarefa é subdividida em subtarefas, que podem ser processadas pela partição em paralelo. Uma vez alocada a um conjunto de nós, o usuário é capaz de iniciar o trabalho em paralelo no forma de subtarefas em qualquer configuração dentro da alocação de nós.

\subsection{cron}

\paragraph{}Outra ferramenta dessa categoria, a utilizada nesse trabalho, é o \textit{crontab}. O nome é uma abreviação de \textit{cron table}, pois se refere ao arquivo \textit{cron file} utilizado internamente para agendamento e execução de tarefas. Esse programa consiste de um arquivo no sistema operacional que periódicamente é lido pelo \textit{cron} daemon, fazendo uso de expressões \textit{CRON}, e suas linhas são interpretadas como entradas de uma tabela contendo em uma coluna a expressão que define sua agenda de execução e em outra coluna o comando a ser executado como um programa de linha de comando. Seu diferencial está no fato de ser extremamente simples e possuir suporte para manipulação por meio de diversas linguagens de programação, incluindo \textit{Python 3}, a linguagem escolhida para o desenvolvimento do sistema aqui detalhado.

\section{Desenvolvimento com Django}

\paragraph{}O \textit{Django} \cite{Django} é um \textit{framework} de desenvolvimento de aplicações e serviços \textit{web} disponível nas linguagens de programação \textit{Python 2 e 3}, gratuito e de código aberto. Sua estrutura interna é baseada no paradigma MTV, particularmente caracterizada por uma forte e nativa integração com bancos de dados a partir de uma interface própria. Todos os componentes de uma aplicação \textit{web} são reproduzidos por meio de convenções próprias ao \textit{framework}, baseados na reusabilidade de código e orientado à integração nativa com bancos de dados diversos.

\paragraph{}O desenvolvimento com o \textit{framework} é marcado pela pouca quantidade de código produzido e pelas diversas interfaces de programação providas pela plataforma. Também se destacam a a integração nativa com um sistema de administração embutido nas aplicações produzidas que fornecem, sem custos adicionais de desenvolvimento, uma aplicação e interface de autenticação de usuários também utilizável em todas as camadas de serviço contruídas paralelamente no mesmo ambiente.

\paragraph{}O \textit{framework} disponibiliza ao desenvolvedor interfaces próprias para o acesso a bancos de dados, templates para renderização de HTML, classes extensíveis para programação orientada a objetos, encaminhamento de requisições configurável, dentre outras facilidades.
